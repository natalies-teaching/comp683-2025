\documentclass{article}

\usepackage{fancyhdr}
\usepackage{extramarks}
\usepackage{amsmath}
\usepackage{amsthm}
\usepackage{amsfonts}
\usepackage{tikz}
\usepackage{url}
\usepackage[plain]{algorithm}
\usepackage{algpseudocode}
\usepackage{hyperref}
\usetikzlibrary{automata,positioning}

%
% Basic Document Settings
%

\topmargin=-0.45in
\evensidemargin=0in
\oddsidemargin=0in
\textwidth=6.5in
\textheight=9.0in
\headsep=0.25in

\linespread{1.1}

\pagestyle{fancy}
\lhead{\hmwkAuthorName}
\chead{\hmwkClass\ (\hmwkTitle}
\rhead{\firstxmark}
\lfoot{\lastxmark}
\cfoot{\thepage}

\renewcommand\headrulewidth{0.4pt}
\renewcommand\footrulewidth{0.4pt}

\setlength\parindent{0pt}

%
% Create Problem Sections
%

%\newcommand{\enterProblemHeader}[1]{
%    \nobreak\extramarks{}{Problem \arabic{#1} continued on next page\ldots}\nobreak{}
%    \nobreak\extramarks{Problem \arabic{#1} (continued)}{Problem \arabic{#1} continued on next page\ldots}\nobreak{}
%}
%
%\newcommand{\exitProblemHeader}[1]{
%    \nobreak\extramarks{Problem \arabic{#1} (continued)}{Problem \arabic{#1} continued on next page\ldots}\nobreak{}
%    \stepcounter{#1}
%    \nobreak\extramarks{Problem \arabic{#1}}{}\nobreak{}
%}
%
%\setcounter{secnumdepth}{0}
%\newcounter{partCounter}
%\newcounter{homeworkProblemCounter}
%\setcounter{homeworkProblemCounter}{1}
%\nobreak\extramarks{Problem \arabic{homeworkProblemCounter}}{}\nobreak{}

%
% Homework Problem Environment
%
% This environment takes an optional argument. When given, it will adjust the
% problem counter. This is useful for when the problems given for your
% assignment aren't sequential. See the last 3 problems of this template for an
% example.
%
\newenvironment{homeworkProblem}[1][-1]{
    \ifnum#1>0
        \setcounter{homeworkProblemCounter}{#1}
    \fi
    \section{Problem \arabic{homeworkProblemCounter}}
    \setcounter{partCounter}{1}
    \enterProblemHeader{homeworkProblemCounter}
}{
    \exitProblemHeader{homeworkProblemCounter}
}

%
% Homework Details
%   - Title
%   - Due date
%   - Class
%   - Section/Time
%   - Instructor
%   - Author
%

\newcommand{\hmwkTitle}{Project Writeup)}
\newcommand{\hmwkDueDate}{April 30, 2025}
\newcommand{\hmwkClass}{Comp683-CompBio}
%\newcommand{\hmwkClassTime}{Section A}
%\newcommand{\hmwkClassInstructor}{Professor Isaac Newton}
\newcommand{\hmwkAuthorName}{\textbf{Your Names Here}} %%modify with your name

%
% Title Page
%

\title{
    \vspace{2in}
    \textmd{\textbf{\hmwkClass\hmwkTitle}}\\
    \normalsize\vspace{0.1in}\small{Due\ on\ \hmwkDueDate\ at 3:10pm}\\
    %$\vspace{0.1in}\large{\textit{\hmwkClassInstructor\ }}
    \vspace{3in}
}

\author{\hmwkAuthorName}
\date{}

\renewcommand{\part}[1]{\textbf{\large Part \Alph{partCounter}}\stepcounter{partCounter}\\}

%
% Various Helper Commands
%

% Useful for algorithms
\newcommand{\alg}[1]{\textsc{\bfseries \footnotesize #1}}

% For derivatives
\newcommand{\deriv}[1]{\frac{\mathrm{d}}{\mathrm{d}x} (#1)}

% For partial derivatives
\newcommand{\pderiv}[2]{\frac{\partial}{\partial #1} (#2)}

% Integral dx
\newcommand{\dx}{\mathrm{d}x}

% Alias for the Solution section header
\newcommand{\solution}{\textbf{\large Solution}}

% Probability commands: Expectation, Variance, Covariance, Bias
\newcommand{\E}{\mathrm{E}}
\newcommand{\Var}{\mathrm{Var}}
\newcommand{\Cov}{\mathrm{Cov}}
\newcommand{\Bias}{\mathrm{Bias}}

\begin{document}

%\maketitle

%\pagebreak
\begin{itemize}
\item Your project is due on canvas by noon on April 30. 
\item You only need to submit one document per group.
\item Feel free to write as much as you need to address the following about your project. For example, a paragraph is probably sufficient for each section. 
\item You do not need to use this LaTeX template, but please ultimately submit a PDF. 
\item Aim to have $\sim$ 2 figures and or tables to report your results. 
\end{itemize}

\section{Title}

What is the title of your project?

\section{Group Members}

Who are the group members working on the project? 

\section{Abstract}

Write a 3-5 sentence summary of the main idea of your project. For example, \\

\emph{Cats are a very common animal on earth. Despite their abundance, the distribution of time they spend sleeping and napping is not well characterized. Here we present DeepCat, a state-of-the-art deep learning approach for learning the transitions of a cat between sleeping and napping. We evaluate our algorithm on three open source cat datasets and achieve superior performance in two out of the three datasets.}

\section{Introduction}

\subsection{Problem Motivation}

Tell us why we care about your problem. 

\subsection{Previous work focused on solving this problem}

Give examples of related work and a high-level understanding of how these methods work.

\subsection{Limitations of previous work}

While previous approaches did X, they could not effectively do Y. 

\section{Statement of Contributions}

In this paper, we developed X. The contributions of X can be summarized as follows.....(You can state computational contributions or applications on dataset contributions). 

\section{Methods}

\subsection{Notation}

We always define our notation!

\subsection{Problem Formulation}

Use mathematical notation here to describe what you are talking about. 

\subsection{Description of your method}

Fill in the details here! 

\subsection{Schematic illustration of your method}

Draw us a `Figure 1` that summarized your method and paper contributions. 

\section{Results}

Overall, aim for 2 (or more!) figures and or tables to illustrate your results. 

\subsection{Datasets}

Describe the details for any datasets you used. 

\subsection{Baselines}

What kinds of baselines did you compare your results to?

\subsection{Description of Experiments}

Describe the experiments (such as metrics that you used to evaluate your results). Point us to the figures or tables that contain these results. 

\section{Discussion}

\subsection{Recap}

Remind the readers what your goal was and what you did. 

\subsection{Observations}

What did your method teach you/ help you to accomplish? How do the results compare to those observed from the baselines?

\subsection{Limitations and Future Work}

What didn't work well? What is still missing? What would you do if you had more time?

\subsection{Inspiring Concluding Paragraph}

Leave us with some inspiration about what your work uncovered and why is is changing the world. 


\end{document}